\chapter{Positional Embedding}

The self-attention mechanism in transformers treats all tokens in a sequence in parallel without an inherent notion of order. This means that, by itself, self-attention is invariant to the order of input tokens. Positional encoding is introduced to inject order information so that the model can differentiate between tokens based on their positions.
\begin{itemize}
	\item \textbf{Absolute Positional Embeddings}: Each position in the sequence is assigned a unique vector. Although straightforward, these embeddings don't scale well to longer sequences and fail to capture the nuances of relative positions between tokens.
	\item \textbf{Relative Positional Embeddings}: These embeddings focus on the distance between tokens, which can improve the model's understanding of token relationships. However, they typically introduce additional complexity into the model architecture.
\end{itemize}

\section{Permutation Invariance of Self-Attention}

Without positional embeddings, the transformer's self-attention would treat the input as a bag of tokens, ignoring the order entirely. The added positional embeddings break this permutation invariance by encoding the position directly into the token representation. As a result, even if the same tokens are present, the model can infer their relative order and roles in the sentence.

Let's look at an example:

\begin{lstlisting}[language=Python]
import torch
import torch.nn as nn
from transformers import AutoTokenizer, AutoModel

# Small, public model (no auth required)
model_id = "prajjwal1/bert-tiny"
tok = AutoTokenizer.from_pretrained(model_id)
model = AutoModel.from_pretrained(model_id)

text = "The dog chased another dog"
tokens = tok(text, return_tensors="pt")["input_ids"]          # [batch, seq]
embeddings = model.get_input_embeddings()(tokens)              # [batch, seq, hidden]
hdim = embeddings.shape[-1]

# Randomly initialized MHA
W_q = nn.Linear(hdim, hdim, bias=False)
W_k = nn.Linear(hdim, hdim, bias=False)
W_v = nn.Linear(hdim, hdim, bias=False)
mha = nn.MultiheadAttention(embed_dim=hdim, num_heads=4, batch_first=True)

with torch.no_grad():
    for param in mha.parameters():
        nn.init.normal_(param, std=0.1)

output, _ = mha(W_q(embeddings), W_k(embeddings), W_v(embeddings))

# With BERT tokenization, sequence is: 
# [CLS], "the", "dog", "chased", "another", "dog", [SEP]
dog1_out = output[0, 2]
dog2_out = output[0, 5]
breakpoint()
print(f"Dog output identical?: {torch.allclose(dog1_out, dog2_out, atol=1e-6)}")
\end{lstlisting}
We use raw token embeddings (no positional encodings). Then, identical tokens (\ie ``dog'') at different positions produce identical attention outputs under our randomly initialized MHA.

If you simply pass this sentence into a transformer, the model wouldn't know the order of the words. The same set of token (\ie word) embeddings could represent a sentence like ``chased dog the dog another'' if no ordering information were provided.

We formulate this as follows: With a permutation matrix \( P \) of shape \( (n, n) \), the input tokens can be permuted as 
\[
X' = P X.
\]
Let's follow the same self-attention process for \( X' \):
\[
Q' = X'W_q = P X W_q = P Q,
\]
\[
K' = X'W_k = P X W_k = P K,
\]
\[
V' = X'W_v = P X W_v = P V.
\]
Then, compute the scores using \( Q' \) and \( K' \):
\[
S' = \frac{Q'(K')^T}{\sqrt{d_k}} = \frac{(P Q)(P K)^T}{\sqrt{d_k}}.
\]
   
Note that
\[
(P K)^T = K^T P^T,
\]
so
\[
S' = \frac{P Q K^T P^T}{\sqrt{d_k}} = P \, S \, P^T.
\]

The softmax is applied row-wise. 
\[
A' = \text{Softmax}(S').
\]
   
Since \( P \) and \( P^T \) are just reordering rows and columns, respectively, the attention weights $A$ are simply permuted like below:
\[
A' = P \, A \, P^T.
\]
Finally, the output for the permuted input is:
\[
\text{Attention}(X') = A' V' = (P A P^T)(P V) = P A V = P \, \text{Attention}(X).
\]
As you can see the self-attention mechanism is \textit{equivariant} to permutations. This means that if you permute the input tokens, the output is permuted in the same way. There is no mechanism in the equations above that distinguishes one ordering from another; the operations treat all tokens symmetrically.

Thus, without additional positional encodings, if you were to shuffle the tokens, the model would compute the same set of pairwise interactions—just in a different order. The structure of the equations does not provide any mechanism for the model to know that one token came before or after another.


\section{Sinusoidal (Absolute) Positional Encoding}

To encode positional information to tokens, there are several desirable properties:
\begin{enumerate}
	\item Unique encoding for each position 
	\item Linear relation between two encoded positions: If we know the encoding for position $p$, it should be straightforward to compute the encoding for position $p+k$, making it easier for the model to learn positional patterns. 
	\item Generalizes to longer sequences: it should generalize outside their training distribution.
	\item Generated by a deterministic process the model can learn: This allows the model to learn the mechanism
	\item Extensible to multiple dimensions: To handle multimodality. 
\end{enumerate}

In Transformer, positional encoding vectors are added to the token (word) embeddings before the input is fed into the self-attention layers. This addition gives the model a sense of the order in the sequence, enabling it to capture the sequential relationships between tokens despite processing them in parallel. The absolute positional encoding is given by
\begin{itemize}
	\item $\text{PE}(pos, 2i) &= \sin\Bigg(\frac{pos}{10000^{\frac{2i}{d_{\text{model}}}}}\Bigg)$
	\item $\text{PE}(pos, 2i+1) &= \cos\Bigg(\frac{pos}{10000^{\frac{2i}{d_{\text{model}}}}}\Bigg)$,
\end{itemize}
where:
\begin{itemize}
	\item \( pos \): the position of the token in the sequence (starting at 0),
	\item \( i \): the index along the embedding dimension,
	\item \( d_{\text{model}} \): the total dimension of the model's embeddings.
\end{itemize}

\begin{figure}[h]
	\centering
	\includegraphics[scale=0.4]{./images/positional_embeddings/ape.png}
	\caption{The visualization of absolute positional embedding. Each dimension oscillates at a different frequency and gets a unique positional encoding.}
\end{figure}

\subsection{Example}
Let's work through a concrete example with a very small embedding dimension:
\begin{itemize}
	\item Assume \( d_{\text{model}} = 4 \).  
	\item We'll compute the positional encoding for two positions: \( pos = 0 \) and \( pos = 1 \).
\end{itemize}

Since \( d_{\text{model}} = 4 \), we have 4 dimensions indexed \(0, 1, 2, 3\). The formulas split the dimensions into even and odd indices:

\paragraph{For \( pos = 0 \)}
\begin{itemize}
\item Dimension 0 (even index, \( i = 0 \)):
 \[
 \text{PE}(0,0) = \sin\Bigl(\frac{0}{10000^{\frac{2\cdot0}{4}}}\Bigr) = \sin\Bigl(\frac{0}{10000^0}\Bigr) = \sin(0) = 0.
 \]
\item Dimension 1 (odd index, \( i = 0 \)):
 \[
 \text{PE}(0,1) = \cos\Bigl(\frac{0}{10000^{\frac{2\cdot0}{4}}}\Bigr) = \cos(0) = 1.
 \]
\item Dimension 2 (even index, \( i = 1 \)):
 \[
 \text{PE}(0,2) = \sin\Bigl(\frac{0}{10000^{\frac{2\cdot1}{4}}}\Bigr) = \sin\Bigl(\frac{0}{10000^{0.5}}\Bigr) = \sin(0) = 0.
 \]

\item Dimension 3 (odd index, \( i = 1 \)):
 \[
 \text{PE}(0,3) = \cos\Bigl(\frac{0}{10000^{\frac{2\cdot1}{4}}}\Bigr) = \cos(0) = 1.
 \]
\item So, the positional encoding for \( pos = 0 \) is:
\[
[0,\, 1,\, 0,\, 1].
\]
\end{itemize}

\paragraph{For \( pos = 1 \)}
\begin{itemize}
\item Dimension 0 (even index, \( i = 0 \)):
\[
\text{PE}(1,0) = \sin\Bigl(\frac{1}{10000^{\frac{2\cdot0}{4}}}\Bigr) = \sin\Bigl(\frac{1}{10000^0}\Bigr) = \sin(1) \approx 0.84147.
\]

\item Dimension 1 (odd index, \( i = 0 \)):
\[
\text{PE}(1,1) = \cos\Bigl(\frac{1}{10000^{\frac{2\cdot0}{4}}}\Bigr) = \cos(1) \approx 0.54030.
\]
\item Dimension 2 (even index, \( i = 1 \)):
\[
\text{PE}(1,2) = \sin\Bigl(\frac{1}{10000^{\frac{2\cdot1}{4}}}\Bigr) = \sin\Bigl(\frac{1}{10000^{0.5}}\Bigr) = \sin\Bigl(\frac{1}{100}\Bigr) = \sin(0.01) \approx 0.00999983.
\]

\item Dimension 3 (odd index, \( i = 1 \)):
\[
\text{PE}(1,3) = \cos\Bigl(\frac{1}{10000^{\frac{2\cdot1}{4}}}\Bigr) = \cos(0.01) \approx 0.99995.
\]
\item Thus, the positional encoding for \( pos = 1 \) is approximately:
\[
[0.84147,\, 0.54030,\, 0.00999983,\, 0.99995].
\]
\end{itemize}

\subsection{Rotation Matrix Aspect}

For each frequency $\omega_i$, the sinusoidal PE at position $p$ uses a pair
\begin{align*}
	\big[\sin(\omega_i p),\ \cos(\omega_i p)\big].
\end{align*}
Think of this as a 2D vector on the unit circle with angle $\theta=\omega_i p$.

If you move from position $p$ to $p+k$, the angle increases by $\omega_i k$. The new pair is obtained by a $2\times 2$ rotation:
\begin{align*}
	\begin{bmatrix}
		\sin(\omega_i (p+k))\\
		\cos(\omega_i (p+k))
	\end{bmatrix} = 
	\underbrace{
		\begin{bmatrix}
			\cos(\omega_i k) & -\sin(\omega_i k)\\
			\sin(\omega_i k) & \ \cos(\omega_i k)
		\end{bmatrix}}_{\text{rotation by angle }\omega_i k}
	\begin{bmatrix}
		\sin(\omega_i p)\\
		\cos(\omega_i p)
	\end{bmatrix}.
\end{align*}
This is just the angle-addition identity written as a matrix multiply. The key is a relative shift $k$ acts as a linear rotation in each $\sin,\cos$ 2D subspace.

The full PE concatenates many such pairs for different $\omega_i$. Collect them into a vector
\begin{align*}
	\mathrm{PE}(p)=\big[\sin(\omega_1 p),\cos(\omega_1 p),\ \sin(\omega_2 p),\cos(\omega_2 p),\ \dots\big].
\end{align*}
A shift by $k$ is then a \textbf{block-diagonal rotation}: 
\begin{align*}
\mathrm{PE}(p+k)=
	\underbrace{\mathrm{diag}\big(R(\omega_1 k),R(\omega_2 k),\dots\big)}_{=:~R(k)}
	\ \mathrm{PE}(p),
\end{align*}
where each block $R(\omega_i k)$ is the $2\times 2$ rotation above.

\begin{itemize}
	\item Relative structure: Because $\mathrm{PE}(p+k)=R(k)\mathrm{PE}(p)$, relative offsets are linearly encoded.
	\item Extrapolation: No learned parameters; positions outside training range still lie on circles $\to$ better length extrapolation than learned embeddings.
	\item Dot-product behavior (vanilla additive PE): In a standard Transformer, we add PE to token embeddings: $x_p = e_p + \mathrm{PE}(p)$. Attention logits involve
		\begin{align*}
		  x_p^\top x_q = e_p^\top e_q + e_p^\top \mathrm{PE}(q) + \mathrm{PE}(p)^\top e_q + \mathrm{PE}(p)^\top \mathrm{PE}(q).
		\end{align*}
  The PE–PE term depends on ($cos(w_i(q-p))$) (via angle differences), which injects relative info, but cross terms mix with content.
	\item RoPE connection: RoPE takes the rotation view further by rotating $Q$ and $K$ in these 2D subspaces, making attention depend on relative positions more cleanly than simple addition.
\end{itemize}

\paragraph{Example} Pick one frequency $\omega=\tfrac{2\pi}{T}$ (period $T$).
If $T=8\Rightarrow \omega=\tfrac{\pi}{4}$, ($p=2$), ($k=3$):
\begin{align*}
	\mathrm{PE}(2)=\big[\sin(\tfrac{\pi}{2}),\cos(\tfrac{\pi}{2})\big]=[1,0].
\end{align*}
Rotation by $\omega k=\frac{3\pi}{4}$:
\begin{align*}
	R=\begin{bmatrix}
		\cos\frac{3\pi}{4}&-\sin\frac{3\pi}{4}\\
		\sin\frac{3\pi}{4}&\cos\frac{3\pi}{4}
	\end{bmatrix}=
	\begin{bmatrix}
		-\frac{\sqrt2}{2}&-\frac{\sqrt2}{2}\\
		\frac{\sqrt2}{2}&-\frac{\sqrt2}{2}
	\end{bmatrix}.
\end{align*}
Then $R[1,0]^\top = \big[-\frac{\sqrt2}{2}, \frac{\sqrt2}{2}\big]^\top$, which equals $\big[\sin(\omega(p+k)),\cos(\omega(p+k))\big]$ as expected, where $\omega(p+k) = 5\pi/4$.

\subsection{Issues of APE}

\begin{itemize}
	\item Poor length extrapolation
		\begin{itemize}
			\item \textbf{Models see a finite range of absolute indices during training}. At inference, new indices (longer contexts) map to unseen $p_i$—distribution shift.
			\item Sinusoidal APE helps a bit (positions are computed, not learned), but periodicity can cause aliasing at very long lengths and still doesn't give a clean relative bias.
		\end{itemize}
	\item Translation non-equivariance
		\begin{itemize}
			\item If you shift the entire sequence by $+k$, absolute indices change, so the same local pattern appears under different $p_i$. The model must relearn invariances that are naturally relative (\eg the next token depends on the previous few tokens, regardless of where they are).
		\end{itemize}
	\item Polluting the semantic information of token embeddings by adding PE. 
	\item No direct encoding of relative distance
		\begin{itemize}
			\item With APE, attention scores don't decompose nicely into a term that is a function of ($i - j$). The model can learn to approximate relative behavior, but it's indirect and fragile.
		\end{itemize}
	\item Fixed or awkward max length
		\begin{itemize}
			\item Learned APE needs a table up to some $L_{max}$. Going beyond it requires interpolation or ad-hoc extension; both can degrade quality.
			\item Even sinusoidal forms often require careful frequency choices and still degrade for out-of-distribution lengths.
		\end{itemize}
	\item Streaming / chunking pain: 
		\begin{itemize}
			\item For long-form or streaming inference, resetting positions per chunk changes absolute indices and can cause mismatches when stitching attention across chunks. Extra engineering (position remapping) is needed.
		\end{itemize}
\end{itemize}


\section{Rotary Position Embedding (RoPE)}

Rather than adding a positional vector to the token embeddings, \textbf{RoPE rotates the embeddings by a position-specific angle.} Think of it as twisting the embedding in space based on its position. For instance, if you have a simple two-dimensional embedding for the word ``dog'', you can imagine its vector being rotated by an angle \(\theta\) if it's the first word, \(2\theta\) if it's the second word, and so on.

For a high-dimensional embedding $\rvq$ (\eg in \(\mathbb{R}^d\)) at position $p$, we create a block diagonal matrix $\mathrm{diag}\big(R(\omega_1 p),R(\omega_2 p),\dots, R(\omega_{d/2} p)\big)$. Then, we can rotate each dimension of $\rvq$:
\begin{align*}
	\begin{bmatrix}
		R_1 & & \\
		    & R_2 & \\
			&  & R_{d/2}
	\end{bmatrix}
	\begin{bmatrix}
		q_1\\
		\vdots\\
		q_d
	\end{bmatrix}
\end{align*}
As you can see, RoPE divides the vector into \(d/2\) pairs (or 2D subspaces). For a token at position \(i\), denote its embedding by:
\begin{align*}
	\mathbf{x}_i \in \mathbb{R}^d.
\end{align*}
We partition \(\mathbf{x}_i\) into pairs:
\begin{align*}
	\mathbf{x}_i^{(k)} = 
	\begin{pmatrix}
	x_{i,2k} \\
	x_{i,2k+1}
	\end{pmatrix},\quad k = 0, 1, \ldots, \frac{d}{2}-1.
\end{align*}
Subsequently, for each 2D subspace indexed by \(k\), RoPE defines a rotation angle:
\[
\theta_{i, k} = i \cdot \alpha_k,
\]
where \(\alpha_k\) is a scaling factor that typically depends on the dimension \(k\). A popular choice is:
\[
\alpha_k = \frac{1}{10000^{\frac{2k}{d}}}.
\]
This scaling mimics the frequency scaling in sinusoidal embeddings (\ie positional embedding), ensuring that different subspaces capture positional information at different granularities.

In two-dimensional geometry, any rotation by an angle \(\theta\) can be represented by a \textit{rotation matrix} \(R(\theta)\). This matrix is a standard tool in linear algebra and has the form:
\[
R(\theta) = 
\begin{pmatrix}
\cos(\theta) & -\sin(\theta) \\
\sin(\theta) & \cos(\theta)
\end{pmatrix}.
\]
We compute its rotated version:
\[
\text{RoPE}_i \bigl(\mathbf{x}_i^{(k)}\bigr)
= R(\theta_{i, k})\, \mathbf{x}_i^{(k)}
= \begin{pmatrix}
\cos(\theta_{i, k}) & -\sin(\theta_{i, k}) \\
\sin(\theta_{i, k}) & \cos(\theta_{i, k})
\end{pmatrix}
\begin{pmatrix}
x_{i,2k} \\
x_{i,2k+1}
\end{pmatrix}.
\]
This yields the updated coordinates:
\[
\begin{aligned}
\tilde{x}_{i, 2k} &= x_{i,2k}\cos(\theta_{i,k}) - x_{i,2k+1}\sin(\theta_{i,k}),\\[6pt]
\tilde{x}_{i, 2k+1} &= x_{i,2k+1}\cos(\theta_{i,k}) + x_{i,2k}\sin(\theta_{i,k}).
\end{aligned}
\]
After processing all \(d/2\) subspaces, we concatenate the results back into a full \(d\)-dimensional vector \(\tilde{\mathbf{x}}_i\).

In transformer architectures, \textbf{RoPE is applied to both the query and the key vectors} except the value vecotrs:
\[
\begin{aligned}
\tilde{\mathbf{q}}_i &= \text{RoPE}_i(\mathbf{q}_i), \\
\tilde{\mathbf{k}}_j &= \text{RoPE}_j(\mathbf{k}_j).
\end{aligned}
\]

Then, the attention score between positions \(i\) and \(j\) is calculated as:
\[
\text{Attention}(i, j) = \frac{\tilde{\mathbf{q}}_i \cdot \tilde{\mathbf{k}}_j}{\sqrt{d}}.
\]

A key property of RoPE is that the dot product between the rotated vectors can be reinterpreted to show how relative positions are encoded. In particular, one can derive that:
\[
\tilde{\mathbf{q}}_i^\top \tilde{\mathbf{k}}_j = \mathbf{q}_i^\top\, \mathbf{M}(i,j)\, \mathbf{k}_j,
\]
where \(\mathbf{M}(i,j)\) is a block diagonal matrix that encapsulates the effect of the relative positional difference \(j-i\).

Focus on one 2D subspace (indexed by \(k\)):
\begin{itemize}
	\item The query subvector at position \(i\) is rotated by \(\theta_{i,k} = i\alpha_k\).
	\item The key subvector at position \(j\) is rotated by \(\theta_{j,k} = j\alpha_k\).
\end{itemize}

The dot product in this subspace is:
\[
\tilde{\mathbf{q}}_i^{(k)\top} \tilde{\mathbf{k}}_j^{(k)} 
= \left(\mathbf{q}_i^{(k)}\right)^\top R(\theta_{i,k})^\top R(\theta_{j,k}) \, \mathbf{k}_j^{(k)}.
\]
Since the transpose of a rotation matrix is its inverse (i.e., \(R(\theta)^\top = R(-\theta)\)), we have:
\[
R(\theta_{i,k})^\top R(\theta_{j,k}) = R(-\theta_{i,k})R(\theta_{j,k}) = R\bigl(\theta_{j,k} - \theta_{i,k}\bigr).
\]
Because \(\theta_{j,k} - \theta_{i,k} = (j-i)\alpha_k\), the transformation becomes:
\[
R\bigl((j-i)\alpha_k\bigr).
\]

Repeating this for each 2D subspace results in a block diagonal matrix:
\[
\mathbf{M}(i,j) = \text{diag}\Bigl(
R\bigl((j-i)\alpha_0\bigr),\,
R\bigl((j-i)\alpha_1\bigr),\,
\ldots,\,
R\bigl((j-i)\alpha_{\frac{d}{2}-1}\bigr)
\Bigr).
\]
Each block is the 2×2 rotation matrix:
\[
R\bigl((j-i)\alpha_k\bigr)
= \begin{pmatrix}
\cos\bigl((j-i)\alpha_k\bigr) & -\sin\bigl((j-i)\alpha_k\bigr) \\
\sin\bigl((j-i)\alpha_k\bigr) & \cos\bigl((j-i)\alpha_k\bigr)
\end{pmatrix}.
\]

\begin{itemize}
	\item Relative Positional Bias:  
  The matrix \(\mathbf{M}(i,j)\) adjusts the dot product between queries and keys based on their relative positions \((j-i)\). Thus, the value vectors are not modified. Instead of explicitly adding a relative position embedding, the rotation inherently modulates the interaction between tokens.
	\item Unified Encoding: Since \(\mathbf{M}(i,j)\) is built from standard rotation matrices \(R(\theta)\), it seamlessly encodes the relative positional difference across all 2D subspaces. This results in a unified treatment where both absolute and relative positional cues are embedded into the attention calculation.
	\item Elegant Mathematical Foundation: The use of \(R(\theta)\) comes directly from classical geometry and linear algebra. It leverages the well-known properties of rotations in 2D—specifically, that rotations preserve vector norms and that the composition of rotations is itself a rotation (with the angle being the sum or difference of the individual angles). This mathematical elegance translates into an efficient and effective mechanism for positional encoding.
	\item Geometric Interpretation: 
		\begin{itemize}
			\item The rotation matrix \(R(\theta)\) rotates any 2D vector by the angle \(\theta\) (in the counterclockwise direction) while preserving its magnitude. 
			\item This property makes it ideal for encoding positional shifts—rotating a vector \textbf{does not change its content (its norm) but changes its direction}, thereby encoding positional information.
		\end{itemize}
	\item Inherent Relative Encoding: When different positions correspond to different rotation angles, \textbf{the relationship between any two positions can be captured by the difference in their rotation angles}. This leads to a natural encoding of relative position without having to explicitly compute or store separate relative position vectors.
\end{itemize}

\section{Position Interpolation (PI)}

If a model is trained with a maximum context length (L) but we want to read sequences up to $L' = sL$, where $s>1$. PI \emph{compresses} absolute indices so that positions never exceed the range seen in training. Concretely, for a token at position $i$ (0-based), PI uses a \emph{scaled} index
\begin{align*}
	\hat{i} = \frac{i}{s}.
\end{align*}
RoPE angles are then computed with $\hat{i}$ instead of $i$. Using the notation from above, each 2D subspace $k$ is rotated by
\begin{align*}
	\hat{\theta}_{i,k} = \hat{i} \alpha_k =
	\frac{i}{s}\alpha_k,
	\quad
	\text{where}\quad
	\alpha_k = \frac{1}{10000^{2k/d}}
\end{align*}
Thus the PI-rotated query/key components become
\begin{align*}
	\widetilde{\mathbf{q}}^{(k)}_i=
	R\left(\frac{i}{s}\alpha_k\right)\mathbf{q}^{(k)}_i,
	\qquad
	\widetilde{\mathbf{k}}^{(k)}_j=
	R\left(\frac{j}{s}\alpha_k\right)\mathbf{k}^{(k)}_j.
\end{align*}
The per-subspace attention contribution depends on the \emph{scaled} relative offset:
\begin{align*}
	\left(\widetilde{\mathbf{q}}^{(k)}_i\right)^\top
	\left(\widetilde{\mathbf{k}}^{(k)}_j\right)=
	\left(\mathbf{q}^{(k)}_i\right)^\top
	R\left(\frac{j-i}{s},\alpha_k\right)
	\mathbf{k}^{(k)}_j.
\end{align*}
Equivalently, PI replaces $j-i$ with $(j-i)/s$ in the block-diagonal matrix $\mathbf{M}(i,j)$. In other words, \textbf{PI uniformly stretches the wavelength} in all RoPE planes by the same factor $s$.

In sum, PI squeezes the new sequence inside the original context window. Desptie its simplicity, PI has several limitations: 
\begin{itemize}
	\item It normally requires finetuning on about billion tokens.
	\item the perplexity slightly increases
\end{itemize}

\section{NTK-Aware Scaled RoPE}

Recall that RoPE encodes positions by rotating each query/key vector by angles that grow with token index. Those angles come from a bank of sinusoidal frequencies $\theta_i = b^{-2i/d}$, where $b$ is the base, typically $10,000$, $d$ is the head dimension, and $i$ indexes each 2D subspace.

During the pre-training, the inputs are squeezed into chunks of sequences of equal amounts of tokens $L$. After the pre-training, it tends to perform badly on sequences longer than $L$. Instead of training again on longer chunks, \textit{NTK-aware scaling} changes the base of RoPE. Empirically this lets you extend context (often 2–4$\times$) with small perplexity hit, even without fine-tuning; with light fine-tuning it goes further. 

Let $s$ be the \textit{extension factor} (\eg going $4k$ $\to$ $16k$ means $s=4$), and let $|D|=d$ be the per-head dimension. NTK-aware scaling replaces the RoPE base $b$ by a larger base $b'$ by
\begin{align*}
	b'=b \cdot s^{\frac{d}{d-2}}
\end{align*}
Then, you recompute RoPE frequencies as usual:
\begin{align*}
	\theta_i' = b'^{-2i/d}\quad i=0\dots \frac{d}{2}-1.
\end{align*}

This is the \textit{Adjusted Base Frequency (ABF)} view derived from NTK theory; it leaves token indices $m$ unchanged and only modifies the frequency bank. 

In short, the trick is to use a larger RoPE base while keeping token indices $m$ unchanged.

\subsection{Intuition}

Increasing $b$ slows the phase growth of rotations at high $i$ (\ie high frequencies), so relative angles between nearby tokens remain similar even when absolute positions are much larger. In the NTK view, that keeps the model's kernel (and thus its inductive bias) closer to what it saw in training, mitigating the usual long-range decay/aliasing of RoPE. Empirical and theoretical works tie context length capability tightly to the RoPE base; too small a base leads to superficial long-context behavior (low perplexity but poor retrieval). 

\begin{lstlisting}[language=Python]
# d_head = per-head dimension (even)
# base = 10000.0 by default
# s = target_context / train_context (e.g., 16_384/4_096 -> 4.0)

def ntk_aware_inv_freq(d_head: int, base: float, s: float):
    import math, torch
    b_prime = base * (s ** (d_head / (d_head - 2)))
    idx = torch.arange(0, d_head, 2.0)
    inv_freq = (b_prime ** (-idx / d_head))  # shape [d_head/2]
    return inv_freq  # use to build cos/sin angles as in standard RoPE
\end{lstlisting}

\section{YaRN: Yet another RoPE extensioN}

Extend context length \emph{without} over-warping local geometry. YaRN does this by \Ni \textbf{interpolating frequencies per dimension} instead of applying a single uniform scale, and \Nii optionally applying a mild \textbf{attention-temperature} adjustment; an \emph{inference-only} dynamic variant is also possible.

\subsection{Dimension-Aware (``by-parts'') Interpolation}

We define $\lambda_k$ as the wavelength of the RoPE embedding at $d$-th hidden dimension:
\begin{align*}
	\lambda_k = \frac{2\pi}{\alpha_k},
\end{align*}
where $k=0,1,\dots,d/2-1$. The wavelength describes the length of tokens needed in order for the RoPE embedding at dimension $k$ to perform a full rotation ($2\pi$).
Let $L$ be the training context and $s = L'/L$ the desired extension factor. We can define a \textit{usage ratio}, which counts how many cycles the $k$-th RoPE plane completes across the training window $L$:
\begin{align*}
r_k \triangleq \frac{L}{\lambda_k}=\frac{L\alpha_k}{2\pi}.
\end{align*}
\begin{itemize}
	\item Small $r_k$ means $\lambda_k \gg L$ (low frequency / long wave). Across the whole training window, this plane barely completes a fraction of a cycle 
	\item Large $r_k$ means $\lambda_k \ll L$ (high frequency / short wave). It oscillates many times inside $L$.
\end{itemize}
Choose thresholds $0<\alpha<\beta$ and a ramp
\begin{align*}
	\gamma (r)= 
	\begin{cases}
	0, & r < \alpha\\
	\frac{r-\alpha}{\beta-\alpha} & \alpha \leq r \leq \beta\\
	1, & r > \beta.
	\end{cases}
\end{align*}
YaRN \textbf{interpolates} each plane's frequency between ``no change'' and a PI-like scaling:
\begin{align*}
	\alpha_k^{\text{(yarn)}} =
	\underbrace{(1-\gamma(r_k))}_{\text{keep low-freq}}
	\cdot \alpha_k \, + \underbrace{\gamma(r_k)}_{\text{scale high-freq}} \cdot \frac{\alpha_k}{s}.
\end{align*}
\begin{itemize}
	\item Leaves low frequencies alone
	\item Scales high frequencies strongly
\end{itemize}
Equivalently, the angle at position $i$ becomes
\begin{align*}
	\theta^{\text{(yarn)}}_{i,k} =
	i \cdot \alpha_k^{\text{(yarn)}} =
	i\cdot\bigg[(1-\gamma(r_k))\alpha_k+\gamma(r_k)\frac{\alpha_k}{s}\bigg].
\end{align*}

Long-wave (low-frequency) planes preserve their original geometry (good for coarse structure), while short-wave planes receive stronger stretching (good for avoiding aliasing at long absolute positions). Mid bands are smoothly blended. Compared with PI, this limits over-compression of frequencies that were already gentle at the training window.

\subsection{Attention-Temperature Adjustment}

YaRN also introduces a softmax temperature $t$ to stabilize attention over longer spans:
\begin{align*}
	\mathrm{Attention}(i,j)=
	\mathrm{softmax}\Big(\frac{\widetilde{\mathbf{q}}_i^\top \widetilde{\mathbf{k}}_j}{t\sqrt{d}}\Big).
\end{align*}
The original paper suggests to use (for $s\geq 1$):
\begin{align*}
	\sqrt{\frac{1}{t}}
	= 0.1 \ln s + 1.
\end{align*}
The $t$ decreases smoothly as the window grows. This mitigates over-sharp attention when many positions are competing at extended lengths.

\subsection{Comparison Summary}
\begin{itemize}
	\item \textbf{PI:} $i\to i/s$ (uniform across planes).
	\item \textbf{NTK-aware base change:} keep $i$ but reduce all frequencies by increasing the RoPE base (global shift). 
	\item \textbf{YaRN:} \emph{per-dimension} blending between NTK-aware and PI-like scaling with mild attention temperature.
\end{itemize}




